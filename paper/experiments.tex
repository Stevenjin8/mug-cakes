
To show that the methods discussed so far work in the real world, we do a quick case study optimizing mug cakes involving human judges.
Baking is an ideal candidate for Bayesian Optimization because trials take a long time since one has to measure, mix, and bake the ingredients.

Gaussian Processes are a perfect model relationship between recipes and their perceived quality because there could be subtle interactions between the recipe as they bake and how they are perceived.
Despite these subtle interactions, we still expect that small in changes in recipe will have little effect on the resulting cake's perceived quality satisfying the continuity assumption of Gaussian Processes.

Additionally, we can incorporate our analysis of noisy and biased observations.
Observations are noisy because the whole baking process, from measuring to baking to tasting, is noisy.
Further, human judges have unique preferences and standards that can bias their results.

\subsection{Recipe and Parameters}

We use a recipe based on \cite{mugcake}, but we vanilla and salt, and have milk be the only wet ingredient.
As such, our five ingredients are flour, sugar, cocoa, and milk.

Our tunable parameters are the volume proportion of flour $x_\ff$, sugar $x_\ss$, cocoa $x_\cc$, and milk $x_\mm$.
We require these proportions to add be positive and sum to 1.
As such, our tunable parameters can be represented by a vector $\mathbf{x} = (x_\ff, x_\ss, x_\cc,x_\mm)$ in a 3-dimensional simplex.
We fix the amount of baking soda to 1/3 teaspoons.
Given parameters, our recipe consists of
\begin{itemize}
    \item $24x_\ff$ teaspoons of all purpose flour
    \item $24x_\ss$ teaspoons of sugar
    \item $24x_\cc$ teaspoons of cocoa powder
    \item $24x_\mm$ teaspoons milk
    \item 1/4 teaspoons of baking soda
\end{itemize}
To bake the cakes,
\begin{enumerate}
    \item Mix dry ingredients in a \textcolor{red}{fixme} mug
    \item Add milk and mix until homogeneous
    \item Place mug on the edge of a 700 Watt microwave and microwave on high for 2 minutes and 15 seconds.
    \item Allow cake to cool for 1 minute.
\end{enumerate}

\begin{figure}[h]
    \centering
    \includegraphics[width=0.33\textwidth]{fig/cake.png}
    \caption{A sample cake.}
    \label{fig:cake}
\end{figure}

\subsection{Setup}

To perform Bayesian Optimization, we use a Gaussian Process surrogate model and $vEI$ as the acquisition function.
For the Gaussian Process, our domain $\mathcal{X}$ was a 3-dimensional simplex, we used mean function was $m(\mathbf{x}) = 3$ for all $\mathbf{x}$, and we used a RBF kernel with parameters $\ell^2 = 0.08^2, \sigma^2_f = 1 ^ 2$.
We incorporated noise and bias as outlined in Section \ref{sssec:post-inf} with a observation noise of $\sigma^2_{\epsilon} = 0.3 ^ 2$ and observer bias variance of $\sigma_b^2 = 0.2$ There were $N_b = 2$ observers.
\textcolor{red}{justify $m$ and $\kappa$ and $\kappa$ params}
\subsection{Procedure}

We follow Algorithm~\ref{alg:bo}.
To observe $f(\mathbf{x})$ for some $\mathbf{x} \in \mathcal{X}$, we baked the recipe as outlined above and had one observer taste the recipe and give it a score out of 10.
Ingredients were measured with measuring spoons with the smallest one being 1/8 teaspoon.
When finding the $\argmax_{\mathbf{x} \in \mathcal{X} } a_{\vei}( \mathbf{x} | \mathcal{D}_n)$, (line 5 of Algorithm~\ref{alg:bo}), ties were broken arbitrarily.

To speed up the process, the first 4 trials were given a score of 0 by observer 1 without actually baking the executing the recipe since they effectively consisted of one ingredient (and baking soda).
We conducted 6 more trials with alternating observers.

\subsection{Inner Optimization Loop}

One practical issues came when executing the inner optimization loop (line 5 of Algorithm~\ref{alg:bo}).
Scipy's Basin Hopping \cite{scipy, wales1997}, uses SciPy's implementation of L-BFGS-B under the hood by default.
However, this implementation of L-BFGS-B only supports domains that are rectangles in the form
\begin{equation*}[a_1, b_1] \times \ldots \times [a_N, b_N]
\end{equation*}
which does not match the form of our domain $\mathcal{X}$ which is a 3-dimensional simplex.
As such, we use surjective differentiable map $g: [0, 1]^3 \to \mathcal{X}$:
\begin{equation*}
    g(\mathbf{r}) =
    \begin{bmatrix}
        r_1 \\
        r_2(1 - r_1) \\
        r_3(1 - r_1 - r_2(1 - r_1)) \\
           1 - r_1 - r_2(1 - r_1) - r_3(1 - r_1 - r_2(1 - r_1)) \\
    \end{bmatrix}
\end{equation*}
to reparametrize $\mathcal{X}$.
Since $g$ has a domain of a rectangle, we can optimize $a_{\vei}(g(\cdot) | \mathcal{D}_n)$ using Scipy's implementation Basin Hopping.
Thus, we can implement the inner optimization loop as
\begin{equation*}
    \argmax_{\mathbf{x} \in \mathcal{X}} a_{\vei}(\mathbf{x} | \mathcal{D}_N)
    =
    g\left(\argmax_{\mathbf{r} \in [0, 1]^3} a_{\vei}(g(\mathbf{r}) | \mathcal{D}_N)\right).
\end{equation*}

\subsection{Results and Discussion}

Figure~\ref{fig:exp-res} shows the result of our experiments.
In Figure \ref{subfig:exp:progression} we see that our algorithm first explores the domain by sampling the corners 4 corners $\mathcal{X}$ and the center.
Once it sees that $\mathbf{x}_5$ did better than the other samples, our algorithm exploits this knowledge by sampling around $\mathbf{x}_5$.
It does not stop exploiting this knowledge because it gets generally increasingly results.
Ultimate, the best recipe was $\mathbf{x}_7$.

With only 10 trials and no prior knowledge, our algorithm was able to find an acceptable recipe validating our method.
One taster described it as ``a bit bland but nothing that destroys your soul or taste buds''.
Further, Figure~\ref{subfig:exp:bias-diff} shows that 0 was no in the 95\% credible interval of $b_1 - b_2$
suggesting that our model for observer bias is effective and can help our optimization algorithms converge faster.

\begin{figure}
    \centering
    \begin{subfigure}[b]{0.31\textwidth}
        \centering
        \includegraphics[width=\textwidth]{fig/progression.png}
        \caption{}
        \label{subfig:exp:progression}
    \end{subfigure}
    \hfill
    \begin{subfigure}[b]{0.31\textwidth}
        \centering
        \includegraphics[width=\textwidth]{fig/ys.png}
        \caption{}
        \label{subfig:exp:ys}
    \end{subfigure}
    \hfill
    \begin{subfigure}[b]{0.31\textwidth}
        \centering
        \includegraphics[width=\textwidth]{fig/bais-diff.png}
        \caption{}
        \label{subfig:exp:bias-diff}
    \end{subfigure}
    \hfill
    \caption{
        Results of Mug Cake case study.
        (a) the progression of ingredient proportions.
        (b) the progression of scores.
        (c) posterior belief about the difference in bias between our observers after all observations.
    }
    \label{fig:exp-res}
\end{figure}
