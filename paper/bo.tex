With a strong understanding of Gaussian Processes, we know begin our discussion of Bayesian
Optimization.
Bayesian Optimization is a iterative algorithm consisting of a surrogate model and an acquisition function.
A surrogate model represents our beliefs about the objective function $f$ given our previous observations 
$\mathcal{D}_n = (\mathbf{x}_1, \ldots, \mathbf{x}_n, y_1, \ldots, y_n, z_1, \ldots, z_n)$
while
an acquisition function $a( \cdot | \mathcal{D}_n) = a_n(\cdot)$.
quantifies how much ``utility'' we get from sampling a point given our beliefs about $f$.
The pseudocode for Bayesian optimization is as follows.

