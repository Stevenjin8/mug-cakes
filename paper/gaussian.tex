A popular way to model an unknown function $f: \mathcal{X} \subseteq \mathbb{R}^{K} \to \mathbb{R}$
is with a finite number of unknown parameters.
For example, with multilinear regression,
we assume that our unknown objective function $f$ is affine and represent our beliefs about $f$ with a random function $F$ in the form
\begin{equation*}
    F(\mathbf{x}) = \bsy{\beta}^{T} \mathbf{x} + \beta_0
\end{equation*}
where $\bsy{\beta} \in \mathbb{R}^{K}$ and $\beta_0 \in \mathbb{R}$ are unknown.
Given some observations, we can perform inference on our unknown parameters
allowing us
to perform inference on $F$.

The problem with parametric methods that they assume that $f$ has a specific shape.
In multilinear regression, the shape is a plane.
However, rarely can we make strong assumptions about $f$'s shape.
If we are, say, a novice baker, then we do not know enough about the baking process and ingredients interactions.
In this section, we review some properties of Gaussian Distributions and extend them to Gaussian processes which will allow us to model our beliefs about $f$ without parametric assumptions.

\subsection{Gaussian Distributions}\label{ssec:gaus}

We begin by defining Gaussian random variables.
\begin{definition}[Nondegenerate Gaussian]
    The probability density of a $K$-dimensional (Multivariate) Nondegenerate Gaussian random vector $\mathbf{y} \sim \mathcal{N}_K(\bsy{\mu}, \bsy{\Sigma})$
    is
    \begin{equation}
        \label{eq:mvn-pdf}
        \mathcal{N}_K(\mathbf{y} | \bsy{\mu}, \bsy{\Sigma}) = \frac{ 1 }{ (2 \pi)^{K/2} \lvert \bsy{\Sigma} \rvert^{1/2} } \exp \left\{ -\frac12 (\mathbf{y} - \bsy{\mu})^{T} \bsy{\Sigma}^{-1} (\mathbf{y} - \bsy{\mu})  \right\}
    \end{equation}
    where $\bsy{\mu} \in \mathbb{R}^{K}$ is the mean vector and $\bsy{\Sigma}$ is a $K \times K$ strictly positive definite covariance matrix.
    We just write $\mathcal{N}$ instead of $\mathcal{N}_K$ if $K$ is clear from context.
\end{definition}
We can also allow $\bsy{\Sigma}$ to be positive semi-definite giving us a possibly degenerate Gaussian, but these are much harder to work with because we rely on $\bsy{\Sigma}^{-1}$ in Equation~\ref{eq:mvn-pdf}.

One reason why Gaussians are popular is because they stay Gaussian under various transformations, as the next theorem shows.

\begin{theorem}[Once Gaussian Always Gaussian]\label{thm:ogag}
    Suppose $\mathbf{y} = (\mathbf{y}_1, \mathbf{y}_2)$ is a $K$-dimensional multivariate Gaussian with mean and variance
    \begin{equation*}
        \bsy{\mu} =
        \begin{bmatrix}
            \bsy{\mu}_1 \\
            \bsy{\mu}_2
        \end{bmatrix}
        ,\quad
        \bsy{\Sigma} =
        \begin{bmatrix}
            \bsy{\Sigma}_{11} &  & \bsy{\Sigma}_{12} \\
            \bsy{\Sigma}_{21} &  & \bsy{\Sigma}_{22} \\
        \end{bmatrix}
    \end{equation*}
    where
    \begin{align*}
        &\mathbf{y}_1, \bsy{\mu}_1 \in \mathbb{R}^{K_1};
        \mathbf{y}_2, \bsy{\mu}_2 \in \mathbb{R}^{K_2}; \\
        &\bsy{\Sigma}_{11} \in \mathbb{R}^{K_1 \times K_1};
        \bsy{\Sigma}_{22} \in \mathbb{R}^{K_2 \times K_2};
        \bsy{\Sigma}_{12} = \bsy{\Sigma}_{21}^{T} \in \mathbb{R}^{K_1 \times K_2}; \\
        &K_1 + K_2 = K.
    \end{align*}
    Then, the following statements are true.
    \begin{enumerate}
        \item
            The marginal of $\mathbf{y}_1$ is
            \begin{equation*}
                \mathbf{y}_{1} \sim \mathcal{N}_{K_1}(\bsy{\mu}_1, \bsy{\Sigma}_{11}).
            \end{equation*}

        \item
            The conditional of $\mathbf{y}_1$ given $\mathbf{y}_2$ is Gaussian
            \begin{align*}
                \mathbf{y}_1 | \mathbf{y}_2 & \sim \mathcal{N}_{K_1}(\bsy{\mu}_{1 | 2}, \bsy{\Sigma}_{1 | 2}) \\
                \bsy{\mu}_{1 | 2} & = \bsy{\mu}_1 + \bsy{\Sigma}_{12} \bsy{\Sigma}_{22}^{-1} (\mathbf{y}_2 - \bsy{\mu}_2) \\
                \bsy{\Sigma}_{1|2 } & = \bsy{\Sigma}_{11} - \bsy{\Sigma}_{12} \bsy{\Sigma}_{22}^{-1} \bsy{\Sigma}_{21}.
            \end{align*}

        \item
            An affine transformation of $\mathbf{y}$ is Gaussian.
            If $\mathbf{A} \in \mathbb{R}^{J \times K}$ and $\mathbf{b} \in \mathbb{R}^{J}$, then
            \begin{equation*}
                \mathbf{Ay} + \mathbf{b} \sim \mathcal{N}_J(
                \mathbf{A} \bsy{\mu} + \mathbf{b},
                \mathbf{A} \bsy{\Sigma} \mathbf{A}^{T}
                ).
            \end{equation*}

        \item
            If $\mathbf{y}' \sim \mathcal{N}_J( \bsy{\mu}', \bsy{\Sigma}')$ and $\mathbf{y}$ are independent, then
            \begin{equation*}
                \begin{bmatrix}
                    \mathbf{y} \\
                    \mathbf{y}'
                \end{bmatrix}
                \sim \mathcal{N}_{K + J} \left(
                \begin{bmatrix}
                        \bsy{\mu} \\
                        \bsy{\mu}'
                    \end{bmatrix}
                ,
                \begin{bmatrix}
                        \bsy{\Sigma} & \mathbf{0}_{K \times J} \\
                        \mathbf{0}_{J \mathbf{x}
                        K} & \bsy{\Sigma}'
                    \end{bmatrix}
                \right).
            \end{equation*}
    \end{enumerate}
\end{theorem}
\begin{proof}
    See Chapter 4 of \cite{murphy2012} for (1)-(3).
    For part (4), suppose that both $\bsy{\Sigma}$ and $\bsy{\Sigma}'$ are nonsingular
    and let
    \begin{equation*}
        \bsy{\Sigma}_t =
        \begin{bmatrix}
            \bsy{\Sigma} & \mathbf{0} \\
            \mathbf{0} & \bsy{\Sigma}'
        \end{bmatrix}
        , \bsy{\mu}_t =
        \begin{bmatrix}
            \bsy{\mu} \\
            \bsy{\mu}'
        \end{bmatrix}
        .
    \end{equation*}
    Then,
    \begin{equation*}
        \bsy{\Sigma}_t^{-1}
        =
        \begin{bmatrix}
            \bsy{\Sigma}^{-1} & \mathbf{0} \\
            \mathbf{0} & (\bsy{\Sigma}')^{-1}
        \end{bmatrix}
        \text{ and }
        \lvert \bsy{\Sigma}_t \rvert
        = \lvert \bsy{\Sigma} \rvert \left\lvert \bsy{\Sigma}' \right\rvert.
    \end{equation*}
    It follows that the density of $\mathbf{t} = (\mathbf{y}, \mathbf{y}')$ is
    \begin{align*}
        &\mathcal{N}_K(\mathbf{y} | \bsy{\mu}, \bsy{\Sigma})
        \mathcal{N}_J(\mathbf{y}' | \bsy{\mu}', \bsy{\Sigma}') \\
        & = \frac{ 1 }{ (2 \pi)^{(K + J)/2} \lvert \bsy{\Sigma} \rvert^{1/2} \lvert \bsy{\Sigma}' \rvert^{1/2} }
        \exp \left\{ -\frac12 (\mathbf{y} - \bsy{\mu})^{T} \bsy{\Sigma}^{-1} (\mathbf{y} - \bsy{\mu})
        -\frac12 (\mathbf{y}' - \bsy{\mu}')^{T} (\bsy{\Sigma}')^{-1} (\mathbf{y}' - \bsy{\mu}')  \right\} \\
        & = \frac{ 1 }{ (2 \pi)^{(K+ J)/2} \lvert \bsy{\Sigma}_t \rvert^{1/2} }
        \exp \left\{ -\frac12 (\mathbf{t} - \bsy{\mu}_t)^{T} \bsy{\Sigma}_t^{-1} (\mathbf{t} - \bsy{\mu}_t)  \right\} \\
        & = \mathcal{N}_{K + J} \left(
            \mathbf{t} | \bsy{\mu}_t, \bsy{\Sigma}_t
        \right).
    \end{align*}
    Since the joint density of $(\mathbf{y}, \mathbf{y}')$ is Gaussian, the joint distribution of $(\mathbf{y}, \mathbf{y}')$ is Gaussian as desired.
\end{proof}
