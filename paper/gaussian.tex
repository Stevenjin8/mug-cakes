Gaussians distributions are one of the most used distributions in statistics due to their pervasiveness in the real world and pleasant properties.
Although we cannot use Gaussian distributions to model unknown continuous functions over an infinite domain, Gaussian distributions will give us the tools to define Gaussian Processes which will serve as our surrogate model.

\subsection{Gaussian Distributions}

We begin by defining Gaussian random variables.

\begin{definition}[Nondegenerate Gaussian]
    The probability density of a $K$-dimensional (Multivariate) Nondegenerate Gaussian random vector $\mathbf{x} \sim \mathcal{N}_K(\bsy{\mu}, \bsy{\Sigma})$
    is
    \begin{equation}
        \label{mvn-pdf}
        \mathcal{N}_K(\mathbf{x} | \bsy{\mu}, \bsy{\Sigma}) = \frac{ 1 }{ (2 \pi)^{K/2} \lvert \bsy{\Sigma} \rvert^{1/2} } \exp \left\{ -\frac12 (\mathbf{x} - \bsy{\mu})^{T} \bsy{\Sigma}^{-1} (\mathbf{x} - \bsy{\mu})  \right\}
    \end{equation}
    where $\bsy{\mu} \in \mathbb{R}^{K}$ is the mean vector and $\bsy{\Sigma}$ is a $K \times K$ strictly positive definite covariance matrix.
    We just write $\mathcal{N}$ instead of $\mathcal{N}_K$ if $K$ is clear from context.
\end{definition}

We can also allow $\bsy{\Sigma}$ to be positive semi-definite giving us a possibly degenerate Gaussian, but these are much harder to work with because we rely on $\bsy{\Sigma}^{-1}$ in Equation~\ref{mvn-pdf}.

One reason why Gaussians are popular is because they stay Gaussian under various transformations, as the next theorem shows.

\begin{theorem}[Once Gaussian Always Gaussian]\label{thm:ogag}
    Suppose $\mathbf{x} = (\mathbf{x}_1, \mathbf{x}_2)$ be a $K$-dimensional multivariate Gaussian with mean and variance
    \begin{equation*}
        \bsy{\mu} =
        \begin{pmatrix}
            \bsy{\mu}_1 \\
            \bsy{\mu}_2
        \end{pmatrix}
        ,\quad
        \bsy{\Sigma} =
        \begin{pmatrix}
            \bsy{\Sigma}_{11} &  & \bsy{\Sigma}_{12} \\
            \bsy{\Sigma}_{21} &  & \bsy{\Sigma}_{22} \\
        \end{pmatrix}
    \end{equation*}
    where
    \begin{align*}
        \mathbf{x}_1, \bsy{\mu}_1 \in \mathbb{R}^{K_1};
        \mathbf{x}_2, \bsy{\mu}_2 \in \mathbb{R}^{K_2};
        \bsy{\Sigma}_{11} \in \mathbb{R}^{K_1 \times K_1};
        \bsy{\Sigma}_{22} \in \mathbb{R}^{K_2 \times K_2};
        \bsy{\Sigma}_{12} = \bsy{\Sigma}_{21}^{T} \in \mathbb{R}^{K_1 \times K_2}.
    \end{align*}
    Then, the following statements are true.
    \begin{enumerate}
        \item
            The marginal of $\mathbf{x}_1$ is
            \begin{equation*}
                \mathbf{x}_{1} \sim \mathcal{N}_{K_1}(\bsy{\mu}_1, \bsy{\Sigma}_{11}).
            \end{equation*}

        \item
            The conditional of $\mathbf{x}_1$ given $\mathbf{x}_2$ is Gaussian
            \begin{align*}
                \mathbf{x}_1 | \mathbf{x}_2 & \sim \mathcal{N}_{K_1}(\bsy{\mu}_{1 | 2}, \bsy{\Sigma}_{1 | 2}) \\
                \bsy{\mu}_{1 | 2} & = \bsy{\mu}_1 + \bsy{\Sigma}_{12} \bsy{\Sigma}_{22}^{-1} (\mathbf{x}_2 - \bsy{\mu}_2) \\
                \bsy{\Sigma}_{1|2 } & = \bsy{\Sigma}_{11} - \bsy{\Sigma}_{12} \bsy{\Sigma}_{22}^{-1} \bsy{\Sigma}_{21}.
            \end{align*}

        \item
            An affine transformation of $\mathbf{x}$ is normal.
            If $\mathbf{A} \in \mathbb{R}^{J \times K}$ and $\mathbf{b} \in \mathbb{R}^{J}$, then
            \begin{equation*}
                \mathbf{Ax} + \mathbf{b} \sim \mathcal{N}_J(
                \mathbf{A} \bsy{\mu} + \mathbf{b},
                \mathbf{A} \bsy{\Sigma} \mathbf{A}^{T}
                ).
            \end{equation*}

        \item
            If $\mathbf{y} \sim \mathcal{N}_J( \bsy{\mu}_y, \bsy{\Sigma}_{y})$ and $\mathbf{x}$ are independent, then
            \begin{equation*}
                \begin{bmatrix}
                    \mathbf{x} \\
                    \mathbf{y}
                \end{bmatrix}
                \sim \mathcal{N}_{K + J} \left(
                \begin{bmatrix}
                        \bsy{\mu} \\
                        \bsy{\mu}_{y}
                    \end{bmatrix}
                ,
                \begin{bmatrix}
                        \bsy{\Sigma} & \mathbf{0}_{K \times J} \\
                        \mathbf{0}_{J \mathbf{x}
                        K} & \bsy{\Sigma}_y
                    \end{bmatrix}
                \right).
            \end{equation*}
    \end{enumerate}
\end{theorem}
\begin{proof}
    See Section 4 of \cite{murphy2012} for (a)-(c).
    For part (d), suppose that both $\bsy{\Sigma}$ and $\bsy{\Sigma}_y$ are nonsingular
    and let
    \begin{equation*}
        \bsy{\Sigma}_{xy} =
        \begin{bmatrix}
            \bsy{\Sigma} & \mathbf{0} \\
            \mathbf{0} & \bsy{\Sigma}_y
        \end{bmatrix}
        , \bsy{\mu}_{xy} =
        \begin{bmatrix}
            \bsy{\mu} \\
            \bsy{\mu}_y
        \end{bmatrix}
        .
    \end{equation*}
    Then,
    \begin{equation*}
        \bsy{\Sigma}_{xy}
        =
        \begin{bmatrix}
            \bsy{\Sigma}^{-1} & \mathbf{0} \\
            \mathbf{0} & \bsy{\Sigma}_y^{-1}
        \end{bmatrix}
        \text{ and }
        \lvert \bsy{\Sigma}_{xy} \rvert
        = \lvert \bsy{\Sigma} \rvert \lvert \bsy{\Sigma}_y \rvert.
    \end{equation*}
    It follows that the density of $\mathbf{t} = (\mathbf{x}, \mathbf{y})$ is
    \begin{align*}
        &\mathcal{N}_K(\mathbf{x} | \bsy{\mu}, \bsy{\Sigma})
        \mathcal{N}_J(\mathbf{y} | \bsy{\mu}_y, \bsy{\Sigma}_y) \\
        & = \frac{ 1 }{ (2 \pi)^{(J + K)/2} \lvert \bsy{\Sigma} \rvert^{1/2} \lvert \bsy{\Sigma}_y \rvert^{1/2} }
        \exp \left\{ -\frac12 (\mathbf{x} - \bsy{\mu})^{T} \bsy{\Sigma}^{-1} (\mathbf{x} - \bsy{\mu})
        + -\frac12 (\mathbf{y} - \bsy{\mu}_y)^{T} \bsy{\Sigma}_y^{-1} (\mathbf{x} - \bsy{\mu}_y)  \right\} \\
        & = \frac{ 1 }{ (2 \pi)^{(J + K)/2} \lvert \bsy{\Sigma}_{xy} \rvert^{1/2} }
        \exp \left\{ -\frac12 (\mathbf{x} - \bsy{\mu}_{xy})^{T} \bsy{\Sigma}_{xy}^{-1} (\mathbf{t} - \bsy{\mu}_{xy})  \right\} \\
        & = \mathcal{N}_{K + J} \left(
        \begin{bmatrix}
                \mathbf{x} \\
                \mathbf{y}
            \end{bmatrix}
        \middle|
        \begin{bmatrix}
                \bsy{\mu} \\
                \bsy{\mu}_{y}
            \end{bmatrix}
        ,
        \begin{bmatrix}
                \bsy{\Sigma} & \mathbf{0}_{K \times J} \\
                \mathbf{0}_{J \mathbf{x}
                K} & \bsy{\Sigma}_y
            \end{bmatrix}
        \right).
    \end{align*}
    Since the joint density of $(\mathbf{x}, \mathbf{y})$ is Gaussian, the joint distribution of $(\mathbf{x}, \mathbf{y})$ is Gaussian as desired..
\end{proof}
