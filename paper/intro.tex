The textbook way to optimize a continuous objective function $f: [a, b] \to \mathbb{R}$ is to
\begin{enumerate}
    \item Find a formula for $f'$.
    \item Find the critical points by solving $f'(c) = 0$ or find where $f'(c)$ does not exist.
    \item Evaluate $f$ at $a$, $b$, and the critical points.
\end{enumerate}
However, this method is unsuitable for many real-world problems because
we might not have a formula for $f$ or $f'$, and $f$ might be expensive to evaluate.
In this paper, we study how Bayesian optimization optimizes unknown continuous functions that are expensive to evaluate and impossible to evaluate precisely.
For example, given a cake recipe $\mathbf{x}$, the only way for a novice baker to learn about
the quality of a recipe $f(\mathbf{x})$ is by baking the recipe and having judges taste the resulting pastry.
Evaluating the quality of a recipe is expensive because baking is slow and laborious.
Further, the evaluation of a recipe's quality is imprecise because there might be small inaccuracies in measurements
and judges could have different biases and standards.
The only structure we require of $f$ is that it be continuous so that we can perform probabilistic inference on $f$'s global behavior given 
a finite number of imprecise observations.
For baking, this requirement means that small changes in recipes cause small changes in the quality of the recipe.
Baking will be the model problem for this paper, but applications also
include finding optimal placements of sensors \cite{capl2017}, sustainable concrete mixtures \cite{ament2023},
and the most profitable slot machines \cite{shahriari2016}.

More formally, we consider a compact space $\mathcal{X} \subseteq \mathbb{R}^{K}$  and an unknown continuous function $f: \mathcal{X} \to \mathbb{R}$.
In the context of baking, $\mathcal{X}$ is a set of vector-encode recipes
and $f$ maps each vector-encoded recipe to its quality.
For example, given $K$ ingredients, each dimension of $\mathcal{X}$ could represent the proportion of an ingredient in a recipe
and $\mathcal{X}$ would be a $(K - 1)$-dimensional simplex.
Our goal is to estimate
\begin{equation}\label{eq:obj}
    \argmax_{\mathbf{x} \in \mathcal{X}} f(\mathbf{x})
\end{equation}
while minimizing the number of imprecise evaluations of $f$.
(Note that the maximum exists by the Extreme Value Theorem).

In Section~\ref{sec:bo}, we will give an overview of Bayesian optimization and introduce its two components: a surrogate regression model and an acquisition function.
In Section~\ref{sec:sm}, we will explore how to use Gaussian processes
to model about our beliefs about $f$ while accounting for noisy and biased observations.
Section~\ref{sec:af} will present an acquisition function that will tell us where to place our observations.
Finally, Section~\ref{sec:exp} will present a small experiment where we optimize mug cake recipes using Bayesian optimization.

For the sake of correctness, we will only consider Borel subsets of $\mathbb{R}^{N}$ denoted $\mathcal{B}_N$.
This restriction is crucial for theoretical purposes, but has almost no practical importance because
 ``any subset of $\mathbb{R}$ that you can write down in a concrete fashion is a Borel set'' \cite{axler2020}.
Further, all integrals will be Lebesgue integrals using the Lebesgue measure on $\mathbb{R}^{N}$.
Lastly, all functions we consider will be measurable.
Pragmatic readers that are unfamiliar with measure theory can treat $\mathcal{B}_N$ as the powerset of $\mathbb{R}^{N}$ and ignore the rest of this paragraph.

