\subsection{Gaussian Processes}

Gaussian Processes are an extension of Gaussian distributions to random continuous functions $f: \mathcal{X} \to \mathbb{R}$.
That is, for each element $\mathbf{x} \in \mathcal{X}$, we have a random variable $f(\mathbf{x})$.
Working with such a distribution might overwhelming, especially if $\mathcal{X}$ has infinitely many elements.
The key to making this problem tractable is that we only consider $f$ at a finite number of points.


\begin{definition}[Gaussian Process]\label{def:gp}
    A Gaussian Process on $\mathcal{X}$ is a random function $f$ parameterized by
    $m: \mathcal{X} \to \mathbb{R}$ and $\kappa: \mathcal{X} \times \mathcal{X} \to \mathbb{R}$
    such that
    \begin{align*}
        \mathbb{E}[f(\mathbf{x})] &= m(\mathbf{x}) \\
        \mathbb{E}[(f(\mathbf{x}) - m(\mathbf{x}))(f(\mathbf{x}') - m(\mathbf{x}'))] &= \kappa(\mathbf{x}, \mathbf{x}').
    \end{align*}
    Then, for any $\mathbf{x}_1, \ldots,\mathbf{x}_n \in \mathcal{X}$,
    \begin{equation*}
        \mathbf{f} \sim \mathcal{N}\left(\mathbf{m}, \mathbf{K}\right)
    \end{equation*}
    where
    \begin{equation*}
        \mathbf{f} = (f(\mathbf{x}_1), \ldots, f(\mathbf{x}_n)), \mathbf{m} = (m(\mathbf{x}_1), \ldots, m(\mathbf{x}_n)), (\mathbf{K})_{ij} = \kappa(\mathbf{x}_i, \mathbf{x}_j)
    \end{equation*}
    
    For this equation to make sense, we require $\kappa$ to generate positive semi-definite $\mathbf{K}$.
    We denote this as $f \sim GP(m, \kappa)$.
\end{definition}

Before we can analyze GP's, we first have to make sure our definition makes sense.
We like there to exist a joint distribution over $\mathbb{R}^{\mathcal{X}}$
for which the marginals of finite number of random variables $\mathbf{f} = (f(\mathbf{x}_1), \ldots, f(\mathbf{x}_n))$ 
is given as in Definition~\ref{def:gp}:
\begin{equation*}
    h_{x_1, \ldots, x_n}(E) = \int_{E} gaussian pdf d \mathbf{x}
\end{equation*}
where $E \subseteq \mathbb{R}^{n}$ is Borel.
(The requirement that $E$ be Borel is only important for measure theoretic purposes and has little practical implication since ``However, any subset of $\mathbb{R}$ that you can write down in a concrete fashion is a Borel set'' \cite{axler2020}).
The existence of such a joint distribution allows us to apply our standard rules of probability to Gaussian Processes and Theorem~\ref{thm:ogag}.
The Kolmogorov Extension Theorem does just this

\begin{theorem}[Kolmogorov Extension Theorem]\label{thm:kol-ext}
    
\end{theorem}

\begin{theorem}\label{thm:gp-const}
    Gaussian Processes are consistent
\end{theorem}
\begin{proof}
    TODO
\end{proof}


With a solid foundation on Gaussian Processes, we can now interpret the parameters of Gaussian Process.
The idea behind Gaussian Processes is that if $\mathbf{x}$ and $\mathbf{x}'$ are similar, then
$f(\mathbf{x})$ and $f(\mathbf{x}')$ should be similar.
The parameter $\kappa$ describes the similarity between two points.
The more similar $\mathbf{x}$ and $\mathbf{x}'$ are, the higher the value of $\kappa(\mathbf{x}, \mathbf{x}')$.
After all, we expect the covariance between $f(\mathbf{x})$ and $f(\mathbf{x}')$
to be high if $\mathbf{x}$ and $\mathbf{x}'$ are similar.
The parameter $m$ tells us our prior expectation of $f$.
Although $m$ can be useful if we have prior knowledge of $f$, it is common to set $m(\mathbf{x}) = 0$ \cite{murphy2012}.
In Figure \ref{fig:gp-sample}, we use a kernel based on Euclidean distance.
As such, we see that points close to each other have similar values (e.g. continuity).
We will later show that this is the case.
%what?

\begin{figure}
     \centering
     \begin{subfigure}[b]{0.45\textwidth}
         \centering
         \includegraphics[width=\textwidth]{fig/gp-sample2d.png}
         \caption{}
         \label{subfig:2d-gp-sample}
     \end{subfigure}
     \hfill
     \begin{subfigure}[b]{0.45\textwidth}
         \centering
         \includegraphics[width=\textwidth]{fig/gp-sample3d.png}
         \caption{}
         \label{subfig:3d-gp-sample}
     \end{subfigure}
     \hfill
    \caption{Samples of a Gaussian Process using a RBF kernel with parameters $\ell^{2} = 0.1^2$ and $\sigma^{2}_f = 1 ^ 2$ with
        $\mathcal{X} = [0, 1]$ (a) and $\mathcal{X} = [0, 1]^2$ (b).
    }
    \label{fig:gp-sample}
\end{figure}

\subsubsection{Posterior Inference on GP}

We now turn our attention to posterior inference.
That is, we want to quantify our belief of $f$ given some observations.
Suppose we observe that 
\begin{equation*}
    \mathbf{f} = (f(\mathbf{x}_1), \ldots f(\mathbf{x}_N)) = \mathbf{y} = (y_1, \ldots, y_N).
\end{equation*}

Then, for some points of interest $\mathbf{v}_{1}, \ldots, \mathbf{v}_{N_{*}}$, the joint distribution of $\mathbf{f}$ and $\mathbf{f}_* = (f(\mathbf{v}_1), \ldots, f(\mathbf{v}_{N_*}))$ is
\begin{equation*}
    \begin{bmatrix}
        \mathbf{f} \\
        \mathbf{f}_{*} \\
    \end{bmatrix} \sim
    \mathcal{N}\left(
    \begin{bmatrix}
        \mathbf{m} \\ \mathbf{m}_{*}
    \end{bmatrix},
    \begin{bmatrix}
        \mathbf{K} & \mathbf{K}_* \\
        \mathbf{K}_*^T & \mathbf{K}_{* *}
    \end{bmatrix}\right)
\end{equation*}
where 
\begin{align*}
    \mathbf{m} &= (m(\mathbf{x}_1), \ldots, m(\mathbf{x}_N)) \\
    \mathbf{m_*} &= (m(\mathbf{v}_1), \ldots, m(\mathbf{v}_{N_{*}})) \\
    (\mathbf{K})_{ij} &= \kappa(\mathbf{x}_i, \mathbf{x}_j) \\
    (\mathbf{K}_*)_{ij} &= \kappa(\mathbf{x}_i, \mathbf{v}_j) \\
    (\mathbf{K}_{* *})_{ij} &= \kappa(\mathbf{v}_i, \mathbf{v}_j)
\end{align*}

Since Gaussian Processes are consistent by Theorem~\ref{thm:gp-const}, it follows from Theorem~\ref{thm:ogag} that our posterior distribution
of $(f(\mathbf{v}_1), \ldots, f(\mathbf{v}_{N_*}))$ given our observations is Gaussian with mean
\begin{equation} \label{eq:noisless-post-mean}
    \mathbf{m}_* + \mathbf{K}_*^T \mathbf{K}^{-1} (\mathbf{y} - \mathbf{m}), 
\end{equation}
and variance
\begin{equation} \label{eq:noiseless-post-var}
    \mathbf{K}_{* *} - \mathbf{K_*}^{T} \mathbf{K}^{-1} \mathbf{K_*}.
\end{equation}

One issue with applying Equations~\ref{eq:noisless-post-mean} and \ref{eq:noiseless-post-var} is that we assume noiseless and unbiased observations of $f$.
This is solemn the case in real-world experiments.
%For example if we bake two cakes with the same recipe twice, it is highly unlikely that we will result in the exact same cake due to a noisy process.
%This is problematic because later on we will see that a GP with an RBF kernel cannot represent data with two different observations of $f(\mathbf{x})$.
As we can see in Figure ~\ref{subfig:noiseless-post}, not accounting for these factors can cause our model to overfit and give us extreme results.

The fact that Gaussians transform nicely (Theorem~\ref{thm:ogag}) allows us to incorporate noise and bias into our model without significant modifications.
We can incorporate bias and noise by imagining that we have $N_b$ observers.
The $j$th observer has a bias of $b_j$ and the $i$th observation is made by the $z_i$th observer.
Then, each observation $y_i$ is
\begin{equation*}
    y_i = f(\mathbf{x}_i) + b_{z_i} + \epsilon_i
\end{equation*}
If we know $\mathbf{z} = (z_1, \ldots, z_n)$ 
and have the following priors
\begin{align*}
    \epsilon_i \sim_{iid} \mathcal{N}(0, \sigma^2_{\epsilon}) \\
    b_j \sim_{iid} \mathcal{N}(0, \sigma_{b}^2)
\end{align*}
then $\mathbf{y} = (y_1, \ldots, y_n)$ is a linear combination of Gaussian random variables
and is Gaussian by Theorem~\ref{thm:ogag}.

Despite these changes, posterior inference is similar.
Suppose $\mathbf{y} = (y_1, \ldots, y_n)$ are some noisy and biased observations of $(f(\mathbf{x}_1), \ldots, f(\mathbf{x}_n))$
and we would like to perform inference on $\mathbf{f}_* = (f(\mathbf{v}_1), \ldots, f(\mathbf{v}_{N_{*}}))$.
Our observations have the same mean of $\mathbf{m} = (m(\mathbf{x}_1), \ldots, m(\mathbf{x}_n))$ as before because $\epsilon_i$ and $b_j$ have mean 0 for all $i$ and $j$.
Assuming that $f$, $\epsilon_i$, $b_j$ are independent for all $i$ and $j$,
the covariance of $y_i$ and $y_j$ is
\begin{equation*}
    \sigma_{ij} = \kappa(\mathbf{x}_i, \mathbf{x}_j) + \mathbb{I}[z_i = z_j] \sigma^2_b + \mathbb{I}[i = j]\sigma^2_e
\end{equation*}
and the covariance of $\mathbf{y}$ is
\begin{equation*}
    \bsy{\Sigma} = \mathbf{K} + \sigma^{2}_e \mathbf{I} + \sigma_b^2 \begin{bmatrix}
        \mathbf{e}_{z_1} & \hdots & \mathbf{e}_{z_n}
    \end{bmatrix}^{ T}
     \begin{bmatrix}
        \mathbf{e}_{z_1} & \hdots & \mathbf{e}_{z_n}
    \end{bmatrix}.
\end{equation*}

Finally, the covariance between $y_i$ and $f(\mathbf{v}_j)$ is
\begin{align*}
    (\mathbf{K}_{*})_{ij}
    &= \mathbb{E}[ (y_i - m(\mathbf{x}_i))(f(\mathbf{v}_j) - m(\mathbf{v}_j)) \\
    &= \mathbb{E}[ (f(\mathbf{x}_i) - m(\mathbf{x}_i))(f(\mathbf{v}_j) - m(\mathbf{v}_j)) \\
    &= \kappa(\mathbf{x}_i, \mathbf{v}_j).
\end{align*}

Since
\begin{equation*}
    \begin{bmatrix}
        \mathbf{y} \\ \mathbf{f}_{*}
    \end{bmatrix}
    =
    \begin{bmatrix}
        y_1 \\
        \vdots \\
        y_n \\
        f(\mathbf{v}_1) \\
        \vdots \\
        f(\mathbf{v}_{N_{*}}) \\
    \end{bmatrix}
    =
    \begin{bmatrix}
        f(\mathbf{x}_1) + b_{z_1} + \epsilon_1 \\
        \vdots \\
        f(\mathbf{x}_n) + b_{z_n} + \epsilon_n \\
        f(\mathbf{x}_*) \\
        \vdots \\
        f(\mathbf{v}_{N_{*}}) \\
    \end{bmatrix}
\end{equation*}
is a linear function of jointly random Gaussians $f(\mathbf{x}_1), \ldots, f(\mathbf{x}_n), f(\mathbf{v}_1), \ldots f(\mathbf{v}_{N_*}), \epsilon_1, \ldots, \epsilon_n, b_1 \ldots, b_{N_b}$,
the random vector $(\mathbf{y}, \mathbf{f}_*)$ is jointly Gaussian
\begin{equation*}
    \begin{bmatrix}
        \mathbf{y} \\
        f(\mathbf{x}_*)
    \end{bmatrix}
    \sim
    \mathcal{N}\left(
    \begin{bmatrix}
            \mathbf{m} \\
            \mathbf{m}_*
        \end{bmatrix}
    ,
    \begin{bmatrix}
            \bsy{\Sigma} & \mathbf{K}_* \\
            \mathbf{K}_*^T & \mathbf{K}_{* *}
        \end{bmatrix}
    \right).
\end{equation*}
Our posterior of $\mathbf{f}_*$ given our observations is Gaussian with mean
\begin{equation*}
    \mathbf{m}_* + \mathbf{K}_*^T \bsy{\Sigma}^{-1} (\mathbf{y} - \mathbf{m})
\end{equation*}
and variance
\begin{equation*}
    \mathbf{K}_{* *} - \mathbf{K}_*^T \bsy{\Sigma}^{-1} \mathbf{K}_*.
\end{equation*}
These equations are similar to Equations \ref{eq:noisless-post-mean} and \ref{eq:noiseless-post-var},
but replacing $\mathbf{K}$ for $\bsy{\Sigma}$.
This difference is crucial because $\mathbf{K}$ might be singular
while $\bsy{\Sigma}$ is will be strictly positive definite when $\sigma_e^2 > 0$ because for any nonzero $\mathbf{c}$,
\begin{align*}
    \mathbf{c}^{T} \bsy{\Sigma} \mathbf{c}
    = \mathbf{c}^{T} \mathbf{K} \mathbf{c}
    + \mathbf{c}^{T} (\sigma^2_e \mathbf{I}) \mathbf{c} 
    +
    \mathbf{c}^{T}
    \begin{bmatrix}
        \mathbf{e}_{z_1} & \hdots & \mathbf{e}_{z_n}
    \end{bmatrix}^{ T}
     \begin{bmatrix}
        \mathbf{e}_{z_1} & \hdots & \mathbf{e}_{z_n}
    \end{bmatrix}
    \mathbf{c} \geq \mathbf{c}^{T} (\sigma^2_e \mathbf{I}) \mathbf{c}
    > 0.
\end{align*}


Figure~\ref{fig:gp-posteriors} shows the effect of incorporating noisy and biased observations.
expand...

\begin{figure}
    \centering
    \begin{subfigure}[b]{0.3\textwidth}
        \centering
        \includegraphics[width=\textwidth]{fig/noiseless-posterior.png}
        \caption{}
        \label{subfig:noiseless-post}
    \end{subfigure}
    \hfill
    \begin{subfigure}[b]{0.3\textwidth}
        \centering
        \includegraphics[width=\textwidth]{fig/noisy-posterior.png}
        \caption{}
        \label{subfig:noisy-posterior}
    \end{subfigure}
    \hfill
    \begin{subfigure}[b]{0.3\textwidth}
        \centering
        \includegraphics[width=\textwidth]{fig/biased-posterior.png}
        \caption{}
        \label{subfig:biased-posterior}
    \end{subfigure}
    \hfill
    \caption{Posteriors of a Gaussian Process given some noisy biased data.
        RBF kernel parameters are $\ell^{2} = 0.1^2, \sigma^2_{f} = 0.3 ^ 2, \sigma^2_b = 1^2$.
        Blue points have a bias of -1 while orange points have a bias of +1.
        Shaded area is 95\% posterior credible interval.
    }
    \label{fig:gp-posteriors}
\end{figure}
