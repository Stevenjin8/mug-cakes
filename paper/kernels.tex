\subsection{Kernels}

Then central idea of Gaussian Processes is that if $\mathbf{x}$ and $\mathbf{x}'$ are similar, then $f(\mathbf{x})$ and $f(\mathbf{x}')$ should be similar.
Kernels give us a way to quantify the similarity between two points by telling us the covariance between $f(\mathbf{x})$ and $f(\mathbf{x}')$.
The more similar $\mathbf{x}$ and $\mathbf{x}'$, the higher $\kappa(\mathbf{x}, \mathbf{x}') = \Cov(f(\mathbf{x}), f(\mathbf{x}')$.
(Note that this is different from the kernel of a functions, the kernel of a distribution, and the kernel of an operating system.)
In this section we explore the Radial Basis Function (RBF) kernel and derive some theoretical results that solidify our analysis from the previous section.

Strictly speaking, a kernel is just a function $\kappa: \mathcal{X} \times \mathcal{X} \to \mathbb{R}$, but it is also usually nonnegative and symmetric \cite{murphy2012}.
For this paper, we will exclusive work with RBF kernels.

\begin{definition}[Radial Basis Function]
    A radial basis function $\kappa_{\rbf}: \mathbb{R}^{k} \times \mathbb{R}^{k} \to \mathbb{R}$
    is given by
    \begin{equation*}
        \kappa_{\rbf}(\mathbf{x}, \mathbf{x}'; \sigma_f^2, \ell^2)
        = \sigma_f^{2} \exp \left\{ - \frac{ \lVert \mathbf{x} - \mathbf{x}' \rVert^2 }{ 2 \ell^2 }\right\}
    \end{equation*}
    for $\sigma_f^2 > 0$ and $\ell^2 > 0$.

    If the parameters are clear from context or not relevant, we omit them and write $\kappa_{\rbf}(\mathbf{x}, \mathbf{x}')$ instead.
\end{definition}
In the context of Gaussian Processes, $\sigma_f^2$ gives us the prior marginal variance of $f(\mathbf{x})$ for all $\mathbf{x}$ because
\begin{equation*}
    \Var(f(\mathbf{x})) = \Cov(f(\mathbf{x}), f(\mathbf{x})) = \kappa_{\rbf}(f(\mathbf{x}), f(\mathbf{x})) = \sigma_f^2.
\end{equation*}
The parameter $\ell^2$ scales the domain by dividing $\lVert \mathbf{x} - \mathbf{x}' \rVert$.
Large values of $\ell^2$ shrink the domain while small values of $\ell^2$ expand the domain.

An important class of kernel are Mercer kernels.
Such kernels allow us to implicitly map our data to a higher (potentially infinite) dimensional inner product space, and work in that potentially infinite dimensional space with a finite amount of computation.
We will use this definition to show prove necessary and desirable properties of RBF kernels which will make them good kernels for Gaussian Processes.

\begin{definition}[Mercer Kernel]
    A kernel $\kappa: \mathbb{R}^{K} \times \mathbb{R}^{K} \to \mathbb{R}$
    is said to be a Mercer Kernel if there exists an inner product space
    $V$ and mapping $\gamma : \mathbb{R}^{K} \to V$ such that
    \begin{equation*}
        \kappa(\mathbf{x}, \mathbf{x}') = \langle \gamma(\mathbf{x}), \gamma(\mathbf{x}') \rangle.
    \end{equation*}
\end{definition}

It might be surprising that RBF kernels are Mercer kernels because that the canonical inner product, the dot product, has a sum in its formula, but the RBF kernel does not.
The next few results utilize infinite dimensional vector spaces and a clever use of the Taylor expansion of $e^{x}$ to show that the RBF kernels is a Mercer kernel.

\begin{definition}[$L^1$]
    The set $L^1$ is the set of sequences of $\mathbb{R}$ whose infinite sum converges absolutely.
    That is
    \begin{equation*}
        L^1 = \left\{ (f_n)_{n=1}^{\infty} \in \mathbb{R}^{\omega} \middle| \sum_{n=1}^{\infty} \lvert f_n \rvert < \infty \right\}.
    \end{equation*}
    For any two $\mathbf{f} = (f_n), \mathbf{g} = (g_n) \in L^1$, their inner product is
    \begin{equation*}
        \langle \mathbf{f}, \mathbf{g} \rangle = \sum\limits_{n=1}^{\infty} f_ng_n.
    \end{equation*}
\end{definition}
\begin{proposition}
    $L^1$ is an inner product space.
\end{proposition}
\begin{proof}
    We need to show that $L^1$ is a vector space and $\langle \cdot, \cdot \rangle$ is an inner product.

    The set $\mathbb{R}^{\omega}$ is a vector space by Example 6.29 of \cite{axler2020}.
    So to show that $L^1 \subseteq \mathbb{R}^{\omega}$ is a vector space, we just need to show that it is a subspace of $\mathbb{R}^{\omega}$.
    \begin{itemize}
        \item \textbf{additive identity}\\
            $\mathbf{0} = (0, 0, \dots)$ is an element of $L^1$ because
            $\sum\limits_{i=1}^{\infty} \lvert 0 \rvert = 0 < \infty$.
        \item \textbf{closed under addition} \\
            Let $\mathbf{f}, \mathbf{g} \in L^1$.
            Then,
            \begin{align*}
                \sum\limits_{i=1}^{\infty} \lvert f_i + g_i \rvert
                \leq \sum\limits_{i=1}^{\infty} \lvert f_i \rvert + \lvert g_i \rvert
                = \sum\limits_{i=1}^{\infty} \lvert f_i \rvert + \sum\limits_{i=1}^{\infty} \lvert g_i \rvert
                < \infty
            \end{align*}
            meaning that $\mathbf{f} + \mathbf{g} \in L^1$.
        \item \textbf{closed under scalar multiplication} \\
            Let $\mathbf{f} \in L^1$ and $\alpha \in \mathbb{R}$.
            Then
            \begin{equation*}
                \sum\limits_{i=1}^{\infty} \lvert \alpha f_i \rvert
                = \lvert \alpha \rvert \sum\limits_{i=1}^{\infty} \lvert f_i \rvert
                < \infty
            \end{equation*}
            meaning that $\alpha \mathbf{f} \in L^1$.
    \end{itemize}
    Thus, $L_1$ is a vector space.

    Now, we show that $\langle \cdot , \cdot \rangle$ is an inner product by Definition 8.1 of \cite{axler2020}.
    \begin{itemize}
        \item \textbf{positivity}\\
            Let $\mathbf{f} \in L^1$.
            Then
            \begin{equation*}
                \langle \mathbf{f}, \mathbf{f} \rangle
                = \sum f_i^2
                \geq 0.
            \end{equation*}
            At the same time, $\mathbf{f}$ must be bounded, so there exists $M \in \mathbb{R}$ such that $M > \lvert f_i \rvert$ for all $i$.
            It follows that
            \begin{equation*}
                \langle \mathbf{f}, \mathbf{f} \rangle
                = \sum\limits_{i=1}^{\infty} f_i^2
                \leq \sum\limits_{i=1}^{\infty}
                M \lvert f_i \rvert < \infty
            \end{equation*}

        \item \textbf{definiteness} \\ Let $\mathbf{f} \in L^1$.
            Then,
            \begin{align*}
                \langle \mathbf{f}, \mathbf{f} \rangle = 0
                \iff
                \sum\limits_{i=1}^{\infty} f_i^2 = 0
                \iff
                \mathbf{f} = \mathbf{0}.
            \end{align*}

        \item \textbf{linearity in first slot} \\
            Let $\mathbf{f}, \mathbf{g}, \mathbf{h} \in L^1$.
            Then,
            \begin{align*}
                \langle \mathbf{f} + \mathbf{g}, \mathbf{h} \rangle
                = \sum\limits_{i=1}^{\infty} (f_i + g_i)h_i
                = \sum\limits_{i=1}^{\infty} f_ih_i + g_ih_i
                = \sum\limits_{i=1}^{\infty} f_ih_i +\sum\limits_{i=1}^{\infty} g_ih_i
                = \langle \mathbf{f}, \mathbf{h} \rangle + \langle \mathbf{g}, \mathbf{h} \rangle.
            \end{align*}

        \item \textbf{symmetry} \\
            Let $\mathbf{f}, \mathbf{g}, \in L^1$.
            Then,
            \begin{equation*}
                \langle \mathbf{f}, \mathbf{g} \rangle
                = \sum\limits_{i=1}^{\infty} f_ig_i
                = \sum\limits_{i=1}^{\infty} g_if_i
                = \langle \mathbf{g}, \mathbf{f} \rangle.
            \end{equation*}
    \end{itemize}
\end{proof}

\begin{theorem}
    \label{thm:rbf-mercer}
    The RBF kernel is a Mercer kernel.
\end{theorem}
\begin{proof}
    This proof is inspired by \cite{shashua2009}.
    For notational simplicity, we only consider the case where $\mathbf{x} \in \mathbb{R}^{1}$.
    Let $\gamma : \mathbb{R}^{1} \to L^1$ be
    \begin{equation*}
        \gamma(\mathbf{x}) = \sigma_f \exp\left\{- \frac{\mathbf{x}^2}{2\ell^2}\right\} \left(\frac{ (\mathbf{x} / \ell)^{n - 1} }{ \sqrt{(n - 1)!} }\right)_{n=1}^{\infty}.
    \end{equation*}
    We can check that the absolute sum of the elements of $\gamma(\mathbf{x})$ converges using the ratio test, which we omit.
    Then, for $\mathbf{x}, \mathbf{x'} \in \mathbb{R}^{1}$,
    \begin{align*}
        \langle \gamma(\mathbf{x}), \gamma(\mathbf{x'}) \rangle
        & = \sigma_f^2
        \exp \left\{ -\frac{\mathbf{x}^2}{2\ell^2} \right\}
        \exp \left\{ -\frac{ \mathbf{x'}^2}{2\ell^2} \right\}
        \sum\limits_{i=1}^{\infty}
        \frac{ (\mathbf{x}/\ell)^{i - 1} (\mathbf{x'}/\ell)^{i - 1}}{ (i - 1)! } \\
        & = \sigma_f^2
        \exp \left\{ -\frac{ \mathbf{x}^2}{2\ell^2} \right\}
        \exp \left\{ -\frac{ \mathbf{x'}^2}{2\ell^2} \right\}
        \exp \left\{ \frac{\mathbf{x} \mathbf{x'}}{ \ell} \right\} \\
        & = \sigma_f^2 \exp \left\{ -\frac{ \lVert \mathbf{x} - \mathbf{x'} \rVert^2 }{ 2\ell^2 } \right\} \\
        & = \kappa_{\rbf}(\mathbf{x}, \mathbf{x'}).
    \end{align*}

    The proof becomes more laborious when considering $\mathcal{X} \subseteq \mathbb{R}^{N}$ because of the multinomial expansion of $\lVert \mathbf{x} - \mathbf{y} \rVert^2$.
    For a complete proof, see Section 4.3.3 of \cite{shashua2009}.
\end{proof}

Using the fact that the RBF kernel is a Mercer Kernel, we can show that it generates positive semi definite covariance matrices.

\begin{lemma}
    Mercer Kernels are positive semi-definite.
\end{lemma}
\begin{proof}
    Suppose that $\kappa: \mathcal{X} \times \mathcal{X} \to \mathbb{R}$ is a Mercer Kernel.
    Then, there exists a inner product space $V$ and function $\gamma: \mathcal{X} \to V$ such that $\kappa(\mathbf{x}, \mathbf{y}) = \langle\gamma(\mathbf{x}), \gamma(\mathbf{y})\rangle$.

    Now, let $\mathbf{x}_1, \dots, \mathbf{x}_N \in \mathcal{X}$, $\mathbf{c} \in \mathbb{R}^{N}$, and
    \begin{equation*}
        \mathbf{K} =
        \begin{bmatrix}
            \kappa(\mathbf{x}_1, \mathbf{x}_1) & \dots & \kappa(\mathbf{x}_1, \mathbf{x}_N) \\
            \vdots & \ddots & \vdots \\
            \kappa(\mathbf{x}_N, \mathbf{x}_1) & \dots & \kappa(\mathbf{x}_N, \mathbf{x}_N) \\
        \end{bmatrix}
        .
    \end{equation*}
    Then,
    \begin{align*}
        \mathbf{c}^{T} \mathbf{K} \mathbf{c}
        & = \langle c_1 \gamma(\mathbf{x}_1) + \dots + c_N \gamma(\mathbf{x}_N), c_1 \gamma(\mathbf{x}_1) + \dots + c_N \gamma(\mathbf{x}_N) \rangle
        \geq 0
    \end{align*}
    by the positivity of inner products.
\end{proof}

\begin{corollary}
    \label{corr:rbf-psd}
    The RBF kernel is positive semi-definite.
\end{corollary}

In fact, if $\mathbf{x}_1, \dots, \mathbf{x}_n$ are distinct, then $\mathbf{K}$ is strictly positive definite as the following result shows.
Thus, the distribution of $(f(\mathbf{x}_1), \dots f(\mathbf{x}_n))$ will be nondegenerate which allows that allows $f(\mathbf{x}_i)$ to take on any value for all $i$.
This property is important because we want to model continuous function and for any distinct $\mathbf{x}_1, \dots, \mathbf{x}_n$ and $y_1, \dots, y_n$ there exists continuous functions that maps $\mathbf{x}_1, \dots, \mathbf{x}_n$ to $y_1, \dots, y_n$ respectively.

\begin{lemma}
    \label{lem:rbf-pd}
    If $\mathbf{x}_1, \dots, \mathbf{x}_n$ are distinct, then the covariance matrix generated $\mathbf{K}, (\mathbf{K})_{ij} = \kappa(\mathbf{x}_i, \mathbf{x}_j)$ by an RBF kernel will be strictly positive definite.
\end{lemma}
\begin{proof}
    Once again, we only consider the case where $\mathcal{X} \subseteq \mathbb{R}^{1}$.
    Without loss of generality, assume that $\sigma_f^2 = \ell^2 = 1$.
    We can write $\kappa_{\rbf}(\mathbf{x}, \mathbf{y}) = \langle \gamma(\mathbf{x}), \gamma(\mathbf{y}) \rangle$
    where $\gamma: \mathcal{X} \to L^{1}$ is given by
    \begin{equation*}
        \gamma(\mathbf{x}) = \exp\left\{-\frac{\mathbf{x}^2}2 \right\} \left(\frac{ \mathbf{x}^{n - 1} }{ \sqrt{(n - 1)!} }\right)_{n=1}^{\infty}
    \end{equation*}
    as in the proof for Theorem~\ref{thm:rbf-mercer}.

    First, we show that if $\mathbf{x}_1, \dots, \mathbf{x}_N$ are distinct, $\gamma(\mathbf{x}_1), \dots, \gamma(\mathbf{x}_N)$ are linearly independent.
    For contradiction, assume that $\gamma(\mathbf{x}_1), \dots, \gamma(\mathbf{x}_N)$ are linearly dependent.
    Then, there exists an $M \in \mathbb{N}, 0 < M < N$ such that $\gamma(\mathbf{x}_1), \dots, \gamma(\mathbf{x}_M)$
    are linearly independent and for some weights $w_1, \dots, w_M \in \mathbb{R}$,
    \begin{equation}
        \label{eq:dep}
        w_1 \gamma(\mathbf{x}_1) + \dots + w_M \gamma(\mathbf{x}_M) = \gamma(\mathbf{x}_{M + 1}).
    \end{equation}
    Since $\gamma$ does not map to $\mathbf{0}$, we know that $M > 0$ and there exists a $w_p \neq 0$, $p \in \mathbb{N}$, $0 < p \leq M$.

    Looking at Equation~\ref{eq:dep} element-wise, we see that for all $q \in \mathbb{N}, q \geq 0$,
    \begin{align*}
        \sum\limits_{i = 1}^{M} w_i \exp \left\{ -\frac12 \mathbf{x}_i^2 \right\} \mathbf{x}_i^{q}
        = \exp \left\{ -\frac12 \mathbf{x}_{M + 1}^2 \right\} \mathbf{x}_{M+1}^{q}
    \end{align*}
    implying that
    \begin{equation*}
        \sum\limits_{i=1}^{M} \mathbf{x}_i w_i \gamma(\mathbf{x}_i)  = \mathbf{x}_{M + 1} \gamma(\mathbf{x}_{M + 1}).
    \end{equation*}

    In the case that $\mathbf{x}_{M + 1} = 0$,
    \begin{equation*}
        \sum\limits_{i=1}^{M} \mathbf{x}_i w_i \gamma(\mathbf{x}_i) = \mathbf{0}.
    \end{equation*}
    We know that $\mathbf{x}_p w_p \neq 0$ because $w_p \neq 0$ and $\mathbf{x}_p \neq \mathbf{x}_M = 0$.
    As such, we have a found a linear combination of $\gamma(\mathbf{x}_1), \dots, \gamma(\mathbf{x}_M)$ with some nonzero weight that equals $\mathbf{0}$ contradicting our assumption that $\gamma(\mathbf{x}_1), \dots, \gamma(\mathbf{x}_M)$ are linearly independent.

    In the case that $\mathbf{x}_{M + 1} \neq 0$,
    \begin{align}
        &\sum\limits_{i=1}^{M} \mathbf{x}_i w_i \gamma(\mathbf{x}_i) = {\mathbf{x}_{M + 1}}\gamma(\mathbf{x}_{M + 1}) \\
        \implies &\sum\limits_{i=1}^{M} \frac{\mathbf{x}_i}{\mathbf{x}_{M + 1}} w_i \gamma(\mathbf{x}_i) = \gamma(\mathbf{x}_{M + 1}). \label{eq:other-dep}
    \end{align}
    Subtracting Equations \ref{eq:dep} and \ref{eq:other-dep},
    \begin{equation*}
        \sum\limits_{i=1}^{M} w_i \left(\frac{\mathbf{x}_i}{\mathbf{x}_{M + 1}} - 1\right) \gamma(\mathbf{x}_i) = \gamma(\mathbf{x}_{M + 1}) - \gamma(\mathbf{x}_{M + 1}) = \mathbf{0}.
    \end{equation*}
    Since $\mathbf{x}_p \neq \mathbf{x}_{M + 1}$, we conclude that
    \begin{equation*}
        w_p\left(\frac{ \mathbf{x}_p }{ \mathbf{x}_{M + 1} } - 1\right) \neq 0.
    \end{equation*}
    Once again, we have found a linear combination of $\gamma(\mathbf{x}_1), \dots, \gamma(\mathbf{x}_M)$ with some nonzero weights that equals $\mathbf{0}$ which also contradicts the assumption that $\gamma(\mathbf{x}_1), \dots, \gamma(\mathbf{x}_M)$ is linearly independent.
    Since we get a contradiction in both cases, we conclude that $\gamma(\mathbf{x}_1), \dots, \gamma(\mathbf{x}_N)$ are linearly independent.

    Now, we use the independence of $\mathbf{x}_1, \dots, \mathbf{x}_N$ to show that $\mathbf{K}$ is strictly positive definite.
    Let $\mathbf{c} \in \mathbb{R}^{N}$ be nonzero.
    Since $\gamma(\mathbf{x}_1), \dots, \gamma(\mathbf{x}_N)$ are linearly independent,
    \begin{equation*}
        \sum\limits_{i = 1}^{N} c_i \gamma(\mathbf{x}_i) \neq \mathbf{0}.
    \end{equation*}
    It follows that
    \begin{align*}
        \mathbf{c}^{T} \mathbf{K} \mathbf{c}
        = \left\langle \sum\limits_{i}^{N} c_i \gamma(\mathbf{x}_i), \sum\limits_{i}^{N} c_i \gamma(\mathbf{x}_i) \right\rangle
        > 0
    \end{align*}
    by the definiteness property of inner products
    Thus, the RBF kernel generates strictly positive definite covariance matrix $\mathbf{K}$ when $\mathbf{x}_1, \dots, \mathbf{x}_N$
    are distinct.
\end{proof}
